\section{PostgreSQL}

PostgreSQL, often refered as Postgres~\cite{hid-sp18-420-PostgreSQL_About}, is an
open source, object-relational database management system. PostgreSQL is free,
extensible and supports cross platform feature. Its source code is available
with open source licence. Postgres was created at UCB by a computer science
professor named Michael Stonebraker~\cite{hid-sp18-420-PostgreSQL_History}.

PostgreSQL runs on all major operating systems. Initially it was designed to run
on UNIX platforms. Now it works on 34 platforms of Linux along with other
platforms such as all Windows versions, Mac OS X and Solaris. It supports text,
images, sounds, video and includes programming interfaces for different
languages such as C, C++, Java, Perl, Python, Ruby, Tcl and Open Database
Connectivity.

PostgreSQL is completely ACID compliant and transactional. It has complete
support for different features such as foreign keys, joins, views, triggers, and
stored procedure~\cite{hid-sp18-420-PostgreSQL_Wiki}. It includes almost all
data types that are used in SQL, such as INTEGER, NUMERIC, BOOLEAN, CHAR,
VARCHAR, DATE, INTERVAL, and TIMESTAMP data type. It also supports storage of binary large
objects, including pictures, sounds, or
video~\cite{hid-sp18-420-PostgreSQL_About}.
