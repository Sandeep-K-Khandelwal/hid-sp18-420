% status: 0
% chapter: TBD

\title{Amazon S3}


\author{Swarnima Sowani}
\affiliation{%
  \institution{Indiana University}
  \streetaddress{Smith Research Center}
  \city{Bloomington} 
  \state{IN} 
  \postcode{47408}
  \country{USA}}
\email{shsowani@iu.edu}


% The default list of authors is too long for headers}
\renewcommand{\shortauthors}{G. v. Laszewski}

\begin{abstract}

Recent increased usage of internet is casing a huge volume of data creation
which creates a growing demand for large-capacity storage solutions. The
advancements in the cloud technologies has yielded new ways of storing,
accessing and managing data. Cloud storage services enable the storage of data
in an inexpensive, secure, fast, reliable and highly scalable manner over the
internet. Amazon S3 is an object storage built to store and retrieve any amount
of data from anywhere such as web sites and mobile apps and corporate
applications. Amazon S3 is the cloud storage that provides storage solution
withquery-in-place functionality which allows user to run powerful analytics
directly on your data at rest in S3.

This paper describes about the basic concepts of Amazon S3 along with its
features, benefits, different usage patterns and the pricing provided by
Amazon.This way, the paper covers the basic information about Amazon S3 and get
the
understanding of how it can be used depending upon requirements.
\end{abstract}

\keywords{hid-sp18-420, AWS, Amazon S3, i526, Simple Storage System}

\maketitle

\section{About Amazon S3}

Amazon S3 is a simple storage service provided by Amazon web services (AWS)
which mainly focuses on a highly-scalable, reliable, and low-latency data
storage infrastructure at low costs~\cite{hid-sp18-420-amazon-S3-FAQ}. Simple
storage Service is a web service provided by Amazon that can be used to store
and retrieve data of any kind and any amount at anytime from anywhere on the
web. It can also be used for static website hosting for different web
applications. Important feature of S3 is that it is available at any point of
time and can be used to store virtually any kind of data in any
format~\cite{hid-sp18-420-amazon-S3-FAQ}.

One of the important features of using S3 is that it offers a highly durable,
scalable, and secure destination for backing up and archiving critical
data~\cite{hid-sp18-420-amazon-S3}. As per AWS documentation, ``Amazon S3 is
designed to deliver 99.999999999\% durability, and it is used to store data for
millions of applications used by market leaders in every
industry''~\cite{hid-sp18-420-amazon-S3}.

Amazon S3 provides versioning capability to provide even further protection for
stored data. It is easy to define lifecycle rules to automatically migrate less
frequently accessed data. User can store any number of objects. Total volume is
unlimited but one object size can range from 0 bytes to 5 terabytes. With
AmazonS3, user needs to pay only for what the usage is. But price vary as per
the
chosen region of S3.



\section{Concepts and Architecture}

Amazon S3 stores data within resources called buckets and data is stored as an
object. Each bucket has its own unique name and this name is unique globally.
This global namespace allows users to have the unique bucket globally which
helps to organize their data and identify the user to be charged for storage
anddata transfers. It also plays an important role in access control and finally
itserves as the unit of aggregation for audit reports.


One bucket can have unlimited number of objects but number of buckets are
limited to 100 per account. Each object consists of two parts as an opaque blob
and metadata which is nothing but a user specified key value pair and
predefinedmetadata such as LastModified date time.

Users can access or create objects based on the access control restrictions
applied to that bucket. Amazon S3 search functionality is limited to the bucket
and is based on the object name only with either full name or any prefix.
Amazondoes not provide content based search functionality for S3.



\section{Key Features}

\paragraph{Simplicity and Flexibility:} 

Amazon s3 is simple to access using web based AWS management console, mobile
app, full REST APIs and AWS SDKs for integration with Mobile SDK, Java, PHP,
Python, \.NET,
Node.js, Ruby. S3 features allows users to take data driven
approach for optimization and management of data efficiently.



\paragraph{Data Durability and Reliability:} 

``Amazon S3 provides durable infrastructure to store important data and is
designed for durability of 99.999999999\% of objects. Your data is redundantly
stored across multiple facilities and multiple devices in each
facility''~\cite{hid-sp18-420-amazon-S3}.

\paragraph{Simple Data Transfer:} 

Data transfer can be made simple using Amazon S3. Amazon provides set of tools
and different options for migration of data and makes it simple and to move
large volume files from Amazon S3. Using S3 transfer
acceleration~\cite{hid-sp18-420-amazon-S3-dataTransfer-FAQ}, it is fast, easy
and secure to
transfer files over long distance. AWS Snowball, AWS Snowball Edge and AWS
Snowmobile are used for large scale data transfers where transferring is made
atone-fifth of the cost of high speed
internet~\cite{hid-sp18-420-amazon-S3-cloud-migration}.

\paragraph{Security and Access Management:} 

Amazon S3 provides multiple mechanisms to provide data security and to control
or monitor data access to user.
\subparagraph{User identities – }

Each user registered to Amazon’s web service gets an identity. These identities
are linked to credit cards to prevent one person to create multiple fake user
accounts. S3 also has a concept of anonymous requests when no information is
associated about user within the request.
\subparagraph{Access control –}

Access controls are given by access control lists (ACL). Each ACL can specify
the access for up to 100 identities.
Below access controls are supported by AWS S3 – 
\begin{itemize}
\item FULLCONTROL– The owner preserves full control over the object with all
permissions.
\item READ– It is given to buckets or users to read the object. 
\item WRITE– It is given to buckets so that user can create objects in the
bucket;
\item READACL– It is given to buckets or objects by which a user can read the
ACL and get the identity of the owner.
\item WRITEACL– It is given to user to change the Access control list. It is
similar to full
user control since the user can assign any right.
\end{itemize}

\subparagraph{Encryption – }

Data can be securely uploaded or downloaded in S3 using SSL encrypted
endpoints.If user chooses to server side encryption, Amazon s3 will
automatically encrypt
the data on every write and decrypt the data on every
read~\cite{hid-sp18-420-amazon-S3-data-encryption}.

\subparagraph{Versioning – }

Amazon S3 provides versioning capability to preserve and restore all the
versions of objects in a bucket. This allows the recovery for any application
failure or wrong action of user or it is available whenever required for
reference. User can configure life cycle
riles~\cite{hid-sp18-420-amazon-S3-lidecycle-rule} to automatically manage the
versioning.

\paragraph{Storage Classes:}

Amazon S3 provides below storage classes depending upon the requirement and use.Amazon S3 Standard – Used for general-purpose storage of frequently accessed
data.
Amazon S3 Standard-Infrequent Access (Standard-IA) – Used for long-lived, but
less frequently accessed data.
Amazon Glacier – Used for low-cost archival data


\paragraph{Query in Place:} 

This is one of the important feature of Amazon which provides suit of tools for
analysing and processing huge amount of data in the cloud faster along with
different ways to integrating existing workflows with Amazon S3. Using Amazon
S3Select users can scan and filter data without retrieving storage which
accelerates the performance and reduces the cost of analytics. It can process
data within an object in storage at rest, and Amazon Athena and Amazon Redshift
Spectrum enable you to run sophisticated analytics directly on data stored in
S3.




\section{Usage Patterns}
There can be multiple ways and multiple use cases in which user can use Amazon
S3 as a storage solutions. Below are some of the common usage patterns of
AmazonS3.
\begin{enumerate}
\item Static web content and media – Amazon S3 is widely used for storing
staticweb content and media for different applications. Each object in the S3
has a
unique HTTP URL to access that object. Hence, this content can be delivered
directly. Amazon S3 can be
used as an origin store for content delivery networks like Amazon CloudFront.
Amazon S3 is suited for hosting web content that requires bandwidth for
addressing extreme spikes in demand. For fast growing websites such as hosting
data intensive content like video or photo sharing websites, Amazon S3 is the
solution as it does not require any additional storage provisioning.
 
\item Host Static Website – Amazon S3 can be used to host static website as it
provides low cost and highly available solution. User can store static HTML
files, images, videos and scripts such as java script easily.

\item Computation and Analytics – Amazon S3 is used as a data store for big
dataanalytics. It is suitable for data that requires large scale analysis like
financial transaction analysis and multimedia transcoding. Since Amazon S3 is
scalable horizontally, user can access data from multiple computing nodes
simultaneously without any constrain of single connection. AWS offers different
services that helps user to manage big data.

\item Backup and Archive – Amazon S3 offers secure and scalable solution to
backup critical data. User can use Amazon glacier to move data from S3 for
backup purpose. Amazon S3 cross origin replication is also available which can
automatically copy objects to S3 buckets in different regions asynchronously.
Lifecycle rules can be defined to migrate less frequent data to S3 standard –
infrequent access and archive to Amazon Glacier. Amazon also provides
versioning capability for S3 data. User can create different versions of data
and hence can keep the backup with older version and new data will get updated
version.
\end{enumerate}

\section{Pricing}

For Amazon S3 storage, user needs to pay only for the storage used. There is no
cost required for setup and no minimum fee applicable. Amazon S3 pricing is
based on the storage in the form of per gb per month, data transfer and number
of requests per month.
``For new user, AWS free tier provides discount which includes 5GB storage,
20000 get requests, 2000 put requests and 15 GB data transfer for
free''~\cite{hid-sp18-420-amazon-S3-pricing}. Amazon S3 provides pricing that
differs based on region.

\begin{acks}

  The author would like to thank Dr.~Gregor~von~Laszewski for his
  support and suggestions to write this paper.

\end{acks}

\bibliographystyle{ACM-Reference-Format}
\bibliography{report} 
